\documentclass[hoptionsi,review,format=sigplan]{acmart}
\usepackage{geometry}                		% See geometry.pdf to learn the layout options. There are lots.
\geometry{letterpaper}                   		% ... or a4paper or a5paper or ...
%\geometry{landscape}                		% Activate for rotated page geometry
%\usepackage[parfill]{parskip}    		% Activate to begin paragraphs with an empty line rather than an indent
\usepackage{graphicx}				% Use pdf, png, jpg, or eps§ with pdflatex; use eps in DVI mode
								% TeX will automatically convert eps --> pdf in pdflatex
\usepackage{minted}
\setminted[haskell]{escapeinside=@@}

\newenvironment{code}
 {\VerbatimEnvironment
  \begin{minted}[fontsize=\small]{haskell}}
 {\end{minted}}

\newenvironment{codex}
 {\VerbatimEnvironment
  \begin{minted}[fontsize=\small]{haskell}}
 {\end{minted}}
 
\newcommand{\hs}{\mintinline[fontsize=\footnotesize]{haskell}}

\newtheorem{theorem}{Theorem}

\title[Traversals of Infinite Structures]{A Predictable Outcome: An Investigation of Traversals of Infinite Structures}
\author{Gershom Bazerman}
\affiliation{%
  \institution{Awake Security}
  \city{San Jose}
  \country{USA}
}

\begin{comment}
\begin{code}
{-# LANGUAGE DeriveFunctor, StandaloneDeriving, UndecidableInstances, PolyKinds, MultiParamTypeClasses, FlexibleInstances, TypeFamilies, RankNTypes, TupleSections, TypeFamilyDependencies #-}

module Predictable where

import Data.Functor.Compose
import Control.Monad.Reader
import Control.Monad.Writer
import Control.Monad.State
import Data.Functor.Classes

\end{code}
\end{comment}

%\date{}							% Activate to display a given date or no date

\begin{abstract}
Functors with an instance of the \hs{Traversable} type class can be thought of as data structures which permit a traversal of their elements. This has been made precise by the correspondence between traversable functors and finitary containers (aka polynomial functors). This correspondence was established in the context of total, necessarily terminating, functions. However, the Haskell language is non-strict and permits functions that do not terminate. It has long been observed that traversals can at times, in practice, operate over infinite lists, for example in distributing the Reader applicative. The result of such a traversal remains an infinite structure, however it nonetheless is productive -- i.e. successive amounts of finite computation yield either termination or successive results. To investigate this phenomenon, we draw on tools from guarded recursion, making use of equational reasoning internal to Haskell. This is done by introducing a formal \hs{Later} modality in the form of an applicative functor, with an associated \hs{Eventually} monad, and introducing a bisimilarity relation on the latter. Using these tools, it transpires that for a traversal to be possibly infinite is less a restriction on the structure being traversed, but rather a restriction on the applicative performing the traversal -- in particular such an applicative must commute with \hs{Later} up to bisimilarity. We term such functors \hs{Predictable} and investigate their structure, providing a range of examples, and showing how this form of analysis allows fine-grained distinctions to be made on the behavior of various operations.
\end{abstract}

\begin{document}
\maketitle

\section{Introduction}
% TODO CITE
Traversable functors, first introduced by McBride and Paterson, provide a sort of "complement" to the Applicative type class. While the Applicative type class picks out functors which have a (closed) lax-monoidal structure, the Traversable type class picks out functors which can distribute over this monoidal structure. We recall below the definitions of Applicative and Traversable which we will be generalizing in this paper:

\begin{codex}
class Functor f => Applicative f where
     pure :: a -> f a
    (<*>) :: f (a -> b) -> f a -> f b      
    
class Functor t => Traversable t where
    sequence :: Applicative f => t (f a) -> f (t a)
\end{codex}

Typically the Traversable type class is given by an operation \hs{traverse} rather than \hs{sequence} but the former is given by \hs{traverse = sequence . fmap} and the latter is conceptually clearer for our purposes.

By results (independently) of [GIBBONS, OCONNOR] it was established that in a total setting, traversable functors correspond to finitary containers, which is to say that any traversable functor is equivalent to one given as the sum of finite products of its underlying element. Recursive structures in the Haskell language are however, not necessarily finitary. For example, the list functor includes lists which have infinite numbers of elements, aka streams. Often, as defined in Haskell, traversals naturally operate on such infinite structures. As a trivial example, \hs{sequence (repeat id) 1}, which makes use of the \hs{Reader} applicative yields an infinite list \hs{[1,1,1...]}.

However, not all applicative functors are conducive to infinite traversals. For example, \hs{sequence (repeat (Just 1))} yields $\bot$.

And more interestingly yet, some applicative functors allow traversals that are only productive (non-bottom) in some \emph{portion} of their results. For example, making use of the \hs{State} monad:

\begin{codex}
runState (sequence (repeat (modify (+1) >> get))) 0
\end{codex}

yields a tuple whose first projection is the infinite list \hs{[1,2,3...]} and whose second projection is $\bot$.

Internal algebraic reasoning about functions like these can be rather tricky, because equational reasoning is usually performed relative to the total, terminating fragment of the Haskell language. This paper will draw on the toolkit of guarded recursion to provide a straightforward way of using equational reasoning to consider properties of terms that may be infinite. Then, making use of this kit, we will describe a class of functors, which we term \hs{Predictable} functors, that are suited to performing infinite traversals.

\section{Equipping Haskell with a Formal Later modality}

In 2000, Nakhano introduced a typed modal logic for self referential formulae which took semantics in a typed lambda calculus with a modality for guarded recursion. In such a setting, types are able to capture when certain data must be evaluated in order for further evaluation to proceed. A term of type \hs{a} which cannot yet be evaluated is given the type \hs{Later a}, expressing that it may be passed around abstractly, but its structure is not yet necessarily available for computation. In such a setting, there is necessarily no general function \hs{forall a. Later a -> a}, nor is there a general fixpoint operator. However, guarded recursion is allowed in the form of a function \hs{lfix :: (Later a -> a) -> a}. This says "if you give provide a function that "embeds" a \hs{Later a} into an \hs{a}, then it can be iterated." Such a function allows a fixpoint because the input function, by construction, cannot evaluate the thing it is given, only manipulate it abstractly. Hence, the \hs{Later} modality makes explicit the rule of thumb that programmers who work with the fixpoint operator have already internalized.

In a system like this, necessarily finite and possibly infinite data are distinguished through the use of the \hs{Later} modality. Consider for example, the types
\begin{code}
data List a = 
     LNil 
   | LCons a (List a)
   
data Stream a = 
     Nil
   | Cons a (Later (Stream a))
\end{code}
The former is necessarily finite, as without a general fixpoint operator, no infinite inhabitant can be constructed. However, the latter is possibly infinite. For example, \hs{lfix (Cons 1)} will produce an infinite stream of ones.

 By an insight of Atkey and McBride, aside from \hs{lfix}, the later modality has exactly the operations of an applicative functor. In particular, we have:
 
 \begin{codex}
 later :: a -> Later a
 lap :: Later (a -> b) -> Later a -> Later b
 \end{codex}
 
Further, in combination these induce functoriality. These operations come with natural intuitions corresponding to the intended meaning of \hs{Later}. "If I know something know, I still know it later. If I will have a function later, and I will have a value later, then later I also will be able to have the result of applying that function to that value."
 
The recognition of \hs{Later} as an applicative functor means that one can program in a fragment of Haskell which enjoys the same properties as a type system with genuine guarded recursion -- i.e. infinite structures can be manipulated in a fashion such that no function yields $\bot$. The recipe for doing so is painfully simple. One simply introduces an abstract datatype equipped with the proper type class instances, and an \hs{lfix} operator, and then does not export its eliminator. 

This is to say one introduces a module supplying the following, and for all code in the fragment we are reasoning about, does not import the constructor \hs{Later}, thus enforcing abstraction.

\begin{code}
newtype Later a = Later a deriving Functor

instance Applicative Later where
  pure = Later
  Later f <*> Later x = Later (f x)

lfix :: (Later a -> a) -> a
lfix f = fix (f . pure)
\end{code}

Further, for all code in sight, one adheres to the discipline of not allowing any direct recursion to occur, and instead using the \hs{lfix} operator uniformly. Working with this sort of discipline without compiler support is not necessarily a good approach for genuine programming in the large. However, for small-scale equational reasoning it suffices, as the condition is straightforward to check by hand.

In the course of this paper, we will refer to terms written with the above discipline (making use of \hs{Later} and \hs{lfix} with no explicit recursion) as guarded recursive terms, and generally to code written with this discipline as in the guarded recursive fragment. Additionally, we will refer to datatypes which make use of the \hs{Later} modality (such as \hs{Stream}) as guarded datatypes, types of code written in the guarded recursive fragment as guarded recursive types, and functions written in this fragment (whether or not they are recursive) as guarded recursive functions.

As an example of working in guarded recursive fragment, we present a function for interleaving possibly infinite streams, as well as merging them with a (truncated) zip. We leave it as an exercise to verify that, e.g., the \hs{reverse} function cannot be written.

\begin{code}
sinterleave :: Stream a -> Stream a -> Stream a
sinterleave = lfix $ \f s1 s2 -> case s1 of
  (Cons x xs) -> Cons x (f <*> pure s2 <*> xs)
  _ -> s2

szip :: Stream a -> Stream b -> Stream (a, b)
szip = lfix $ \f s1 s2 -> case (s1, s2) of
  (Cons x xs, Cons y ys) -> Cons (x,y) (f <*> xs <*> ys)
  _ -> Nil
\end{code}


\section{Predictable functors and infinite traversals}

With the above tools in hand, we are ready to provide an informal introduction to the central idea of this paper. The forward implication of the correspondence between traversable functors and finitary containers is simply the observation that traversable functors are closed under arbitrary sums and finite products [CITE JASK]. The goal is to extend traversals to handle guarded types such as \hs{Stream}. Hence, we need to close "infinite-traversable" functors under sum, product, and also composition with \hs{Later}. This amounts to having (an appropriately lawful) function \hs{Later (t (f a)) -> f (Later (t a))}. But this decomposes into the composition of \\ \hs{fmap sequence :: Later (t (f a)) -> Later (f (t a))} with a function \hs{predict :: Later (f a) -> f (Later a)}. This latter function makes no mention of \hs{t} and so it is effectively not a property of traversable functors, but rather of the applicative functors used to traverse them. This motivates defining a type class, \hs{Predictable} as follows:

% TODO jask has arb sum, finite product. but we have arb sub, sort of arb product? need better grasp on guarded datatypes

\begin{code}
class Predictable f where
  predict :: Later (f a) -> f (Later a)
\end{code}

And, in turn, we can now define a candidate type class for infinite-traversable functors:

\begin{code}
class ITraversable t where
  isequence :: (Applicative f, Predictable f) => 
              t (f a) -> f (t a)
\end{code}

Since these classes are for reasoning in the guarded recursive fragment, we require that their instances be given only using tools from that fragment. Further, we require three laws analagous to those for standard traversable functors. Namely:

\begin{itemize}
\item \textbf{Identity}: \\
\hs{isequence . fmap Identity} = \hs{Identity}
\item \textbf{Composition}: \\
\hs{isequence . fmap Compose} = \\ \hs{Compose . fmap isequence . isequence}
\item \textbf{Naturality}: \\
\hs{t . sequence} = \hs{fmap isequence . t}
\end{itemize}

where \hs{t} is a natural transformation between predictable applicative functors, which is to say that it commutes with applicative operations and "weakly" commutes with \hs{predict} (i.e. \hs{t . predict } is equivalent to \hs{predict . fmap t} in a sense that will be made precise in the next section).

As an example, here is the \hs{ITraversable} instance for \hs{Stream}.

\begin{code}
instance ITraversable Stream where
  isequence = lfix $ \rec x -> case x of
     Nil -> pure Nil
     Cons a s -> 
         Cons
           <$> a 
           <*> predict (rec <*> s)
\end{code}

Syntactically, it looks nearly the same as the \hs{Traversable} instance for lists, but it makes a judicious use of \hs{predict} to align the \hs{Later} uses in the course of the traversal. As such, straightforward equational reasoning, essentially no different than that for the \hs{Traversable} instance for lists, serves to verify that it satisfies the laws.

In general, for any strictly positive recursive datatype, one can construct a related, "potentially infinite" guarded datatype by syntactically guarding each recursive occurrence by \hs{Later}. Since any \hs{Traversable} datatype can be written as such a strictly positive type, then it follows that for every \hs{Traversable} datatype there is a recursively guarded variant which can be given an instance of \hs{ITraversable}.

As a further example of this, we give the a binary tree with labeled leafs, as well as the recursively guarded version of it, and a corresponding \hs{ITraversable} instance for the latter:

\begin{code}
-- "normal" tree
data Tree a = Leaf a | Branch (Tree a) (Tree a)

-- guarded tree
data ITree a = 
     ILeaf a 
   | ITBranch (Later (ITree a)) (Later (ITree a))

-- corresponding infinite traversable instance
instance ITraversable ITree where
  isequence = lfix $ \rec x -> case x of
     ILeaf a -> ILeaf <$> a
     ITBranch x y -> 
             ITBranch 
          <$> predict (rec <*> x)
          <*> predict (rec <*> y)
\end{code}

The next question to consider is when an applicative functor may be given a valid \hs{Predictable} instance. However, this requires specifying what a "valid" predictable instance is. In turn, to do so, some new tools must be developed. The general idea is that our use of types to track guarded recursion in a fragment of Haskell is merely a way of "annotating" existing code to make it make sense. As such, the function \hs{predict} is intended to be merely an accounting device, witnessing some property of the underlying functor, and not to do actual computation. So we might wish to require it be an isomorphism on types. However, in obviously desirable cases it cannot have an inverse. For example, it is easy to send \hs{Later (a -> b)} to \hs{a -> Later b}, but there is no general function going in the other direction. Therefore, we need a notion of equivalence between terms of heterogenous type that suffices to capture the intended semantics -- for this, we introduce something we term bisimilarity by evaluation.

\section{Bisimilarity by evaluation}
% TODO can dive right in to Later a -> a
As a general motivation, we note that \hs{Later} is not a monad, and in particular, there is no function of the form:
\begin{codex}
forall a. Later (Later a) -> Later a
\end{codex} 

Intuitively, if such a thing existed, it would "collapse" all future timesteps into a single timestep. As such, it would allow unguarded recursion as long as it occurred under at least a single \hs{Later}. 

Nonetheless, we wish to use the guarded recursion modality to reason about terms in Haskell -- a nonstrict language with general recursion. As such, we want to consider equivalence between terms of type \hs{Later a} and \hs{Later (Later a)}, for example. The appropriate notion of equivalence between terms in a nonstrict setting, like that between concurrent terms, should be some form of bisimilarity. In essence, we wish terms to be considered equivalent if under some sequence of abstract observations, each contains the same data, with the same causal ordering dependency. 

Put another way, the reason we do not want a morphism from \hs{Later (Later a)} to \hs{Later a} is because this would allow a use of \hs{lfix} to perform unbounded recursion. However, for purposes of reasoning about equivalence, rather than calculation, such a map is reasonable. In fact, we can go further. For purposes of reasoning up to equivalence, a map \hs{Later a -> a} is reasonable as well. In fact, in general, since \hs{Later} is a newtype, there will be an associated map from any type with \hs{Later} involved to one without \hs{Later} involved, which, considered purely as a Haskell type, will be an isomorphism. The general idea is that terms may defined in a total setting where there is no general map \hs{Later a -> a} and all recursion is via \hs{lfix}. However, bisimilarity between terms is calculated by "evaluating" these terms to a nonstrict setting with general recursion, and then calculating equivalence using the standard tools of denotational semantics. Put another way: rather than constructing a theory of bisimilarity on guarded recursive types intended to capture when they are equivalent in a nonstrict, recursive setting, we simply construct a procedure for evaluating such types to a nonstrict, recursive setting, and directly reason about their equivalence using tools that already exist.

Continuing the thread of reasoning internally in Haskell, we represent evaluation by means of a type class with an associated type:

\begin{code}
class EvalLater a where
  type Result a 
  leval :: a -> Result a
\end{code}

The most important instance is a recursive instance that strips away an outer \hs{Later} and proceeds to continue to evaluate the result. Additionally, all ground types we wish to consider must be equipped with an appropriate instance, and instances for products, sums, and soforth all arise very mechanically. Importantly, this code is not part of the guarded recursive fragment we are reasoning about, but rather belongs to the equational metatheory, and thus freely pattern matches on \hs{Later}. Some example instances are as follows.

\begin{code}
instance EvalLater a => EvalLater (Later a) where
  type Result (Later a) = Result a
  leval (Later x) = leval x

instance EvalLater Int where
  type Result Int = Int
  leval x = x

instance (EvalLater a, EvalLater b) => EvalLater (a, b) 
  where
    type Result (a, b) = (Result a, Result b)
    leval (x, y) = (leval x, leval y)
\end{code}

As we noted early on, when working in the guarded recursive fragment of the language, recursive structures which do not make use of \hs{Later} are already necessarily finite. As such, evaluation on them remains a purely formal operation. For example, for lists we have:

\begin{code}
instance EvalLater a => EvalLater [a] where
  type Result [a] = [Result a]
  leval xs = fmap leval xs
\end{code}

We will refer to structures which do not make use of \hs{Later} either directly or indirectly as \emph{finite} with regards to the guarded recursive fragment.

The only slight complication arises in the case of guarded datatypes, which may themselves make use of \hs{Later}. In such a case, the need for an associated type becomes very clear, as we are evaluating to an entirely different type, which we have established is isomorphic, rather than to a type obviously syntactically related to the given type. For example, partially infinite streams may be evaluated like so:

\begin{code}
instance EvalLater a => EvalLater (Stream a) where
  type Result (Stream a) = [Result a]
  leval Nil = []
  leval (Cons x  xs) = leval x : leval xs
\end{code}

In general, this construction allows that every recursively guarded variant of a strictly positive datatype can evaluate to the original type.

Functors all enjoy the property that \hs{leval = lt . fmap leval}, where \hs{lt} is an appropriate natural transformation. As such while \hs{Result} is only defined on things of kind \hs{*} we also will refer to \hs{Result f} where \hs{f} is a functor. Finite functors enjoy the further property that \hs{lt} is \hs{id}.

We may now define two terms \hs{x} and {y} of types {a} and {b} as \emph{evaluated-bisimilar}  when there is an equivalence \\ \hs{f :: Result a -> Result b} and further \hs{f (leval x)} = \hs{leval y}. For our purposes, this equivalence of evaluated terms will typically be \hs{id}. 

There is thus a sense in which guarded recursive types may be seen as fibered over equivalence classes of standard Haskell types, providing for each such class a (partially ordered) set of possible refinements of evaluation dependency. For example, the types \hs{Stream a}, \hs{PStream a}, \hs{Later [a]}, and \hs{Later (Later [a])} all "live above" \hs{[a]}. We should note that evaluation is defined on all terms, not just those in the guarded-recursive fragment, and consequently evaluated-bisimilarity as a relation can extend not only to guarded recursive terms, but to general terms.

We call a guarded recursive function \hs{a -> b} a guarded weak bi-equivalence (\emph{gwbeq}) if it sends a term to a bisimilar term. The name is justified because all identities are clearly gwbeq and further, it is straightforward to show that the class of gwbeqs is closed under composition and furthermore satisfies "two out of three" -- that is if  \hs{f :: a -> b} and \hs{g :: b -> c} and two out of three of $\{f, g, g \odot f\}$ are gwbeq, then the third is as well. It is important to note that being a gwbeq is substantially weaker than being an isomorphism -- a gwbeq \hs{a -> b} will typically not have a corresponding inverse gwbeq \hs{b -> a} -- though in the next section, we will consider the special case where this does occur.
 
 % probably satisfies two out of six, but should we bother
We will also at times consider bi-equivalences which are not necessarily written in the guarded recursive fragment, and will refer to them as evaluated bi-equaivalences (\emph{ebeq}). By definition, \hs{leval} itself is an ebeq.

\section{Productivity and Stable values}
It is tempting to describe working in a guarded recursive system as a situation "where the types ensure everything you write is productive" -- but this is a misleading intuition. The types ensure that no pattern match on any term will yield $\bot$. However, it is possible to encode data that in the "evaluated" semantics absolutely does yield $\bot$ -- for example, an infinite nesting of \hs{Later}. The point is that, in doing so, the lack of productivity can be read off directly from the types. So guarded recursion is a situation where reasoning about productivity is made manifest at the type level.

More generally, it is useful to to work with types which are only possibly productive -- i.e. where there may be arbitrary sequences of \hs{Later} applications. These can be captured by a \hs{Delay} type, following Capretta.

\begin{code}
data Delay a = Now a | Wait (Later (Delay a))
\end{code}

Using this type, one can, for example write a function to compute the last element (should it exist) of a possibly infinite stream.

\begin{code}
slast :: Stream a -> Delay (Maybe a)
slast = go Nothing
  where go = lfix $ \f def s1 ->
         case s1 of
          (Cons x xs) -> Wait $ f <*> pure (Just x) <*> xs
          Nil -> Now def
\end{code}

This type can also be used to capture structures that are possibly productive, such as possibly productive infinite streams:

\begin{code}
data PStream a =
      PNil
    | PWait (Later (PStream a))
    | PCons a (PStream a)
\end{code}

Corresponding \hs{EvalLater} instances can also be written for possibly productive structures, such as below:

\begin{code}
instance EvalLater a => EvalLater (Delay a) where
  type Result (Delay a) = Result a
  leval (Now x) = leval x
  leval (Wait x) = leval x

instance EvalLater a => EvalLater (PStream a) where
   type Result (PStream a) = [Result a]
   leval PNil = []
   leval (PWait x) = leval x
   leval (PCons x xs) = leval x : leval xs
\end{code}

As discussed above, a gwbeq, unlike an isomorphism, may not be invertible -- i.e. there is always a map \hs{a -> Later a} and there is no general guarded recursive map in the other direction, either as an actual retract, or even as a gwbeq. Possibly productive structures correspond to cases where such a gwbeq \emph{does} exist. We term types that permit such a map "Later-stable" and equip them with a type class as follows:

\begin{code}
class Stable a where
   wait :: Later a -> a

instance Stable (Delay a) where
   wait = Wait

instance Stable (PStream a) where
   wait = PWait
\end{code}

As a convention, we will term types which may be inhabited by infinite nestings of \hs{Later} with no other intervening data (i.e. stable types) as "stable types," and types which nowhere in them contain such nestings as "productive types". A type such as \hs{(Int, Delay Int)} is neither stable nor productive -- it is not stable because it cannot contain \emph{solely} an infinite nesting of \hs{Later}, but it is also not productive, because it may contain such a nesting at some place within it.

As an aside, a gwbeq sending a term to a term that is of a stable type may be termed a "stabilization" and consequently, for any type \hs{a}, \hs{Now :: a -> Delay a} is the universal (terminal) stabilization. 

\section{Properties of predictable functors}
We can now state the desired law of a predicable functor -- the guarded recursive function \hs{predict :: Later (f a) -> f (Later a)} must be a guarded weak bi-equivalence.

Here, we consider some examples, and identify a large class of valid instances. As the instances below show, \hs{Predicable} is closed under composition, and product. Base instances that generate it include exponentiation by a constant, and \hs{Later} itself.
\begin{code}
instance Predictable Later where
  predict = id

instance Predictable ((->) r) where
  predict x = \y -> fmap ($ y) x
   
instance (Predictable f, Predictable g, Functor f) => 
               Predictable (Compose f g) where
  predict = Compose . fmap predict 
               . predict . fmap getCompose

data Prod f g a = Prod {pr1 :: f a, pr2 :: g a}

instance (Predictable f, Predictable g) => 
               Predictable (Prod f g) where
  predict x = Prod 
                  (predict (fmap pr1 x)) 
                  (predict (fmap pr2 x))
\end{code}

Strikingly, \hs{Later} is not closed under sum. In particular, consider the type:

\begin{codex}
data Choice a = C1 a | C2 a
\end{codex}

The outermost constructor of the result of \hs{predict} would necessarily be \hs{C1} or \hs{C2} -- however, by the construction of the \hs{Later} modality, there is no mechanism in the guarded recursive fragment to determine which one. This gives us the inspiration for the name -- a \hs{Predictable} functor is one where, without inspecting the structure, we can "predict" the outermost constructor. In GHC argot, a predictable functor is one where it is safe to preform an irrefutable pattern match.

However, this restriction does not mean that \hs{Predictable} structures are not closed under multiplication by any constant -- in fact, they remain closed precisely under multiplication by \emph{specific} constants -- those which are stable. In particular:

\begin{code}
instance (Stable c) => Predictable ((,) c) where
   predict x = (wait (fst <$> x), snd <$> x)
\end{code}

This turns out to be a bi-implication, and it is easy to show that \hs{((,) c)} is predictable if and only if \hs{c} is stable. In particular, if \hs{Later (c, a)} has a gwbeq to \hs{c, Later a} then so too does \hs{(Later c, Later a)}, and this means that \hs{predict} implies a morphism \hs{Later c -> c} that is a gwbeq. The last is, by definition, stability.

% Show how cont is a predictable functor, but it doesn't matter

\section{Bisimilarity by evaluation for arrow types}

% Establish laws for itraversable

The constructions thus far describe properties holding in the guarded recursive fragment. However, an important goal is to  be able to extract reasoning from this fragment and apply it to general terms, and in particular to establish not only that \hs{isequence} can be defined productively, but the conditions under which a \hs{sequence} is productive. To do so we need to extend further our toolkit for reasoning, and in particular, to extend the notion of evaluated bisimilarity to cover types which contain arrows. When we wish to consider evaluation on the function space, the need arises to go in the other direction as well -- that is, to \emph{lift} general values (which are posited to satisfy appropriate productivity conditions) back into the guarded recursive realm. We therefore introduce the following type class to represent inductively the desired inverse: 

\begin{code}
class EvalLater a => LiftLater a where
  llift :: Result a -> a
\end{code}

We require that \hs{llift} be an ebeq -- i.e. that it act as an inverse (up to bisimilarity) of \hs{leval}.

By recursive definition, the \hs{EvalLater} class allows for a meaningful notion of evaluation on arrow types as follows:

\begin{code}  
instance (LiftLater a, EvalLater b) => 
  EvalLater (a -> b) where
     type Result (a -> b) = Result a -> Result b
     leval f = leval . f . llift

instance (LiftLater a, LiftLater b) => 
   LiftLater (a -> b) where
     llift f = llift . f . leval
\end{code}

Inverses (up to bisimilarity) for various defined evaluations follow mechanically. As some examples:

\begin{code}
instance (LiftLater a) => LiftLater (Later a) where
  llift = Later . llift

instance (LiftLater a) => LiftLater (Delay a) where
  llift = Now . llift

instance (LiftLater a, LiftLater b) => 
   LiftLater (a, b) where
     llift (x, y) = (llift x, llift y)

instance LiftLater Int where
  llift = id
  
instance LiftLater a => LiftLater (Stream a) where
  llift [] = Nil
  llift (x:xs) = Cons (llift x) . Later $ llift xs
\end{code}

It is important to note that these inverses do not, and cannot, send general Haskell terms to guarded recursive terms which are guaranteed productive. In particular, a guarded recursive construction of infinitely nested \hs{Delay} is perfectly reasonable, and under evaluation it goes to $\bot$. Lifting this back in turn does not land in guarded recursive terms (since such terms do not directly contain bottom) but rather in terms which make use of \hs{Later} but exist in general Haskell semantics -- so the lifting in a sense only goes "halfway" back. However, this halfway back intermediate form is all that is necessary for the bookkeeping of the ultimate goal, which is simply that \hs{leval} be defined properly on function types.

In the following sections, we will assume that all types in sight have been equipped with appropriate \hs{LiftLater} and {EvalLater} instances. Further, we note that computing with \hs{LiftLater} is reasonable, but due to injectivity constraints in GHC's checker, \hs{EvalLater} is a class for reasoning with, but much less so computing with.

It is important to note that the ability to evaluate functions does not mean that all algebraic properties of a guarded recursive function necessarily descend to its evaluation. Loosely speaking, \hs{leval} is not necessarily functorial, and \hs{leval f (leval x)} need not be equal to \hs{leval (f x)}. For example, consider a function of type \hs{Delay Int -> Bool} that returns \hs{True} if the outermost constructor is \hs{Now} and \hs{False} otherwise. The evaluation of this function would always yield \hs{True}. The problem is that in a sense the function can detect "how evaluated" its input is. The class of functions that do commute with \hs{leval} are those which do not distinguish between "layers" of \hs{Later} -- i.e. which take bisimilar inputs to bisimilar outputs. We call such functions bisimilar-invariant, and  bisimilar-invariant properties of bisimilar-invariant functions descend to their evaluations, as one would desire.

% TODO leval of a function that is not bisimilar-invariant is still definitionally an ebeq, but somehow does not respect it -- specify how

\section{Predictable functors yield productive traversals}

% TODO define "contains bottom" on guarded recursive fragment, 

% productivity in Later (causal) does permit bottom in a sense -- but it is a step indexed bottom.

% Stable elements do not evaluate to necessarily not-containing bottom things.

% Full theorem holds for finite predictable applicatives -- products (which are also exponentiation by a constant) and Nothing else
% More limited for predictable productive applicatives -- productive is there is always an algebra (f a -> a) for all a. Delay is not productive.
% Cont is _not_ productive
% Actually holds if we require an algebra for all a on the guarded recursive fragment.
% Maybe just require strict positivity, show weird example.
% cont is not productive but cont over writer is productive?

% Need to be more precise with what "productive" means -- means doesn't get stuck, doesn't mean you can't define thigns that don't return a usable value. Just... the evaluation is step indexed. "causal"

We set out to establish when one can "infinite-traverse" a functor productively. Let us now define productive traversal more precisely. Productiveness is typically only defined on functions yielding coinductive structures, and is the property that any finite sequence of operations on the result of such a function will terminate in finite time. However, for a traversal \hs{t (f a) -> f (t a)}, the applicative functor \hs{f} that is the action of the traversal is not necessarily a coinductive structure. Further, it may yield something with a productive \hs{t a} but coupled with a non-terminating component. So some subtlety is required in defining productivity. For the purposes of this paper, we say that a traversal \hs{t} of type \hs{t (f a) -> f (t a)} is productive if \hs{f} has an algebra \hs{alg :: f a -> a}, and for an \hs{x} that does not contain $\bot$, no finite sequence of pattern matches on \hs{alg (t x)} yields $\bot$.

Further, we call a guarded traversal productive if the algebra is itself in the guarded fragment, and further, for an \hs{x} that is productive (can contain no infinite sequence of \hs{Later}), then \hs{alg (t x)} is likewise productive. by the properties of guarded recursion, we can "read off" the productivity of a guarded traversal simply from checking that \hs{t} is itself productive. It is straightforward to check that if a bisimulation-invariant guarded traversal \hs{t} is productive in the guarded recursive sense, then the evaluated traversal \hs{leval t} is productive in the sense we introduced above.

Earlier, we sketched how any \hs{Traversable} functor generates a recursively guarded variant which is an \hs{ITraversable}. Here, we consider the converse. Given any \hs{ITraversable} functor \hs{t}, then for any predictable applicative functor \hs{f}, there is a function \hs{seq :: Result t (f a) -> f (Result t a)}, given by \hs{leval isequence}. Because \hs{leval} is an ebeq, then one can check that the \hs{ITraversable} laws descend to the appropriate \hs{Traversable} laws. At a high level, everything in sight is either removing or adding \hs{Later}, and since \hs{Later} is invisible under \hs{leval} and all functions we consider are bisimulation-invariant, then properties established in the guarded recursive fragment hold under evaluation. Notably, among these properties is productiveness itself. Thus, we can conclude that all \hs{Traversable} functors admit \hs{Traversable} instances which are productive on predictable applicatives.

As a fully worked example, we consider traversals of streams by finitary functors, reasoning equationally, first fusing the evaluation of the result of the sequencing, and then fusing the input to the sequencing.

The initial step makes use of the fact that \hs{f} is a finitary functor. We then substitute the definition of \hs{lfix}. Then, we push the application of \hs{leval} through the case statement, and make use of the fact that \hs{predict}, \hs{llift} and \hs{leval} are all ebeq. Finally, we compose with \hs{llift} which amounts to substitution of pattern matching, changing the type of the recursive call, and elimination of the inner \hs{leval}.

\begin{codex}
sequenceS :: (Applicative f, Predict f) =>
           Stream (f a) -> f (Stream a)
sequenceS = lfix $ \rec x -> case x of
     Nil -> pure Nil
     Cons a s -> 
         Cons 
           <$> a 
           <*> predict (rec <*> s)
           
leval . sequenceS :: Stream (f a) -> f [a]

= fmap leval . fix $ \rec x -> case x of
     Nil -> pure Nil
     Cons a s -> 
         Cons  
           <$> a 
           <*> predict (pure rec <*> s)
           
= fix $ \rec x -> fmap leval $ case x of
     Nil -> pure Nil
     Cons a s -> 
         Cons  
           <$> a 
           <*> predict (fmap llift . rec <$> s)

= fix $ \rec x -> case x of
     Nil -> pure []
     Cons a s -> 
         (:) 
           <$> a 
           <*> leval (rec <$> s)
           
leval . sequenceS . llift :: f [a] -> f [a]

= fix $ \rec x -> case x of
     [] -> pure []
     (:) a s -> 
         (:) 
           <$> a 
           <*> rec s

\end{codex}

This last term is the standard \hs{sequence} for lists, and hence we can conclude that the \hs{Traversable} instance for lists is productive on predictable applicatives (when they have the corresponding algebra necessary to state the property).

Conversely, we can show that if a functor \hs{f} can commute with all \hs{ITraversable} functors, then that functor is necessarily predictable. In particular, \hs{Later} is itself straightforwardly \hs{ITraversable}. Instantiating \hs{isequence} at \hs{Later} yields \hs{Later (f a) -> f (Later a)} -- precisely the type of \hs{predict}. By identity and naturality, we can conclude that this must be gwbeq, which shows the function has the desired properties.

Thus we see that being \hs{Predictable} is both necessary and sufficient for an \hs{Applicative} functor to yield infinite traversals.

% Delay and Guard are an adjoint monad/comonad pair
% make Guard not strong using the Static trick

\section{Examples (and nonexamples) of predictable functors and their traversals}

\begin{comment}
\begin{code}
instance Semigroup w => Semigroup (Delay w) where
  Now x <> Now y = Now (x <> y)
  x <> Wait y = Wait $ fmap (x <>) y
  Wait x <> y = Wait (fmap (<> y) x)

instance Monoid w => Monoid (Delay w) where
  mempty = Now mempty
  
\end{code}
\end{comment}

We now revisit the examples from the introduction, showing how the machinery developed can aid in reasoning about infinite traversals. As a general theme, there are multiple guarded recursive types which lie over every standard type, which may be viewed as "causal refinements" of the underlying type. While all predictable types yield productive traversals, not all capture equally granularly the full properties of the underlying applicative which the are modeling, and there are examples which show that there is not necessarily a "single best" refinement for all uses.

As a warm up, the \hs{Reader} monad is straightforwardly a predictable functor, and so is productive on infinite traversals.

\begin{code}
instance Predictable (Reader r) where
  predict x = reader $ \r -> fmap (($ r) . runReader) x
\end{code}

The \hs{Writer w} monad introduces a complication -- to specify a \hs{Predict} instance, the monoid carried by the writer must be stable:.

\begin{code}
instance (Stable w, Monoid w) => 
         Predictable (Writer w) where
  predict x = writer $ 
      (fst . runWriter <$> x, 
       wait $ snd . runWriter <$> x)
\end{code}

As an example, when \hs{w} is a monoid, \hs{Delay w} has a natural monoid structure and so we have the fact that in general, \hs{runWriter . sequence} (of type \hs{[Writer w a] -> ([a], w)}) will yield a nonbottom second projection only on the occasion that the input list is finite.

However, for specific choices of \hs{w} there are more precise types possible. For instance, there is a straightforward Haskell instance of monoid for \hs{PStream a} -- the type of possibly productive streams.

\begin{code}
instance Semigroup (PStream a) where
   PNil <> x = x
   PCons x xs <> y = PCons x (xs <> y)
   PWait xs <> y = PWait ((<>y) <$> xs)
  
instance Monoid (PStream a) where
   mempty = PNil
\end{code}

The resultant \hs{Writer (PStream a)} expresses that when the accumulator of a writer is itself a list, then sequencing an infinite list can yield a "semi-productive" second component -- in particular, if there are $m$ writes within the first $n$ terms of the list, then sequencing up to the first $n$ terms of the list will yield $m$ results in the second component.

The \hs{State} monad presents difficulties similar to those of \hs{Writer}, but slightly more complicated. The most naive choice for a \hs{Predict} instance would be \hs{State (Delay s)}. But then at any individual point in a computation, the "real" state would be guarded by a \hs{Delay} operator and so appear not necessarily productively accessible. The property we want to capture is more subtle -- on an infinite list, the second projection of \hs{runState} is indeed bottom, as there is no "final" state. However, at any individual point in a \hs{State} computation, the state \emph{thus far} is immediately accessible. A more granular type to capture the productivity of \hs{State} as well as its algebraic operations can be achieved by moving to Ahman and Uustalu's and algebraic generalization of state into the pair of a reader and writer monads -- aka the \hs{Update} monad. The update monad decouples the information on the left and right hand side of the function arrow -- one still reads the state, but writes are not directly of the state, but instead of a monoidal output which \emph{acts} on the state. This decoupling solves the difficulties in presenting a state monad that has good \hs{Predictable} properties -- only the output need be stable, and not the state itself. This does not yield the \hs{State} monad per-se, but does provide a reasonable simulation therein.

We recall briefly the \hs{Update} monad below, and present the appropriate \hs{Predictable} instance.

\begin{code}
data Update p s a = Update {runUpdate :: s -> (p, a)}
    deriving Functor
    
class (Monoid p) => ApplyAction p s where
   applyAction :: p -> s -> s

instance (ApplyAction p s) => Monad (Update p s) where
  Update u >>= f =
    Update $ \s ->
      let (p, a) = u s
          Update t = f a
          (p', a') = t (applyAction p s)
       in (p <> p', a')

putAction :: p -> Update p s ()
putAction p = Update $ \_ -> (p, ())

getState :: Monoid p => Update p s s
getState = Update $ \s -> (mempty, s)

instance Stable p => Predictable (Update p s) where
  predict x = Update $ 
      \s -> predict (($ s) . runUpdate <$> x)
\end{code}

With this in hand, we can give apply actions to appropriate stable structures. For example, the "state-logging" monad can be instantiated as the action of the heads of partially productive streams.

\begin{code}  
instance ApplyAction (PStream a) a where
  applyAction (PCons x _) _ = x
  applyAction _ s = s
\end{code}  

By using "head" instead of "last", unlike the version in Ahman and Uustalu (example 5), this allows infinite traces. The result of an infinite traverse is a pair, whose first component is the semi-productive stream of all intermediate states, and whose second component is the necessarily productive stream of every traversed result. There is an important sublety to this argument -- the instance given for \hs{PStream} is not bisimulation invariant! However, it is invariant in the lucky circumstance that the only actions which are applied are those whose first element is not delayed. Fortunately, the infinite traversals of \hs{Stream} and \hs{ITree} we have considered thus far obey this property, which we term being \emph{prompt}, and will consider further in the next section. At this point, we consider it an open question whether a guarded-recursive and predictable version of \hs{State} is possible such that all its operations can be written in a bisimulation-invariant fashion, though, from the discussion in the next section, it seems unlikely.

% CITE MONADS for Behavior https://www.sciencedirect.com/science/article/pii/S1571066113000650

Finally, we consider a few nonexamples. As one would hope, applicative functors such as \hs{Maybe} and \hs{Either a} cannot be given a \hs{Predictable} instance, which corresponds to the fact that every element of an infinite sequence would need to be inspected before the head of a traversal could be produced -- i.e. a \hs{Nothing} anywhere in an infinite stream would render the entire traversal to be \hs{Nothing}. More subtly, there is the problem of the continuation monad. The continuation monad in fact admits a predictable instance, as given below:

\begin{code}
data Cont r a = Cont {runCont :: (a -> r) -> r}

instance Stable r => Predictable (Cont r) where
  predict x = Cont $ \z -> 
      wait $ fmap ($ (z . pure)) (runCont <$> x) 
\end{code}

However, there is no general algebra \hs{alg :: Cont r a -> a}, and further, even when \hs{r} and \hs{a} are the same type, the stability requirement on \hs{r} and the action of \hs{predict} (which always inserts a wait over the entire structure) means that the result of an infinite traversal is necessarily infinitely delayed. So this gives a situation where an infinite traversal is well-defined in the guarded-recursive fragment, but this traversal is nevertheless not productive in the guarded-recursive sense or in the sense of evaluated semantics.

   
\begin{comment}
\begin{code}
  
foo :: [(Update (PStream Int) Int Int)  ]
foo = [getState, putIt 1 >> getState, putIt 3 >> getState, getState >>= \x -> putIt (x + 1) >> getState, getState, getState >>= \x -> putIt (x + 1) >> getState, getState >>= \x -> putIt (x + 1)  >> getState >>= \x -> putIt (x + 1)  >> getState >>= \x -> putIt (x + 1)  >> getState]
   where putIt i = putAction $ PCons i PNil
  
bar :: [State Int Int]
bar = [get, put 1 >> get, put 3 >> get, get >>= \x -> put (x + 1) >> get, get, get >>= \x -> put (x + 1) >> get, get >>= \x -> put (x + 1) >> get >>= \x -> put (x + 2) >> get]

listToStream (x:xs) = Cons x . Later $ listToStream xs
listToStream [] = Nil

streamToList Nil = []
streamToList (Cons a (Later b)) = a : streamToList b

deriving instance Show a => Show (PStream a)

deriving instance Show a => Show (Stream a)

deriving instance Show a => Show (Later a)

instance (ApplyAction p s) => Applicative (Update p s) where
  pure a = Update $ \_ -> (mempty, a)
  x <*> y = x `ap` y
\end{code}
\end{comment}

\section{Prompt traversals, sequencing, and bi-infinite structures}

A consequence of the analysis of traversals in [Forwards and Backwards] is that traversals of a datatype correspond to permutations of its elements and in fact there are as many traversals of a given datatype as there are permutations on its arities. There is a sense in which this holds true on infinite traversals, but also a sense in which it breaks down. In particular, it holds somewhat true only as long as all functions in sight are bisimulation-invariant. However, as soon as any function (such as the action of the update "simulation" of \hs{State} above) breaks bisimulation-invariance, then the traversal of a structure making use of such functions no longer necessarily corresponds to a traversal in the evaluated semantics. At first, this may seem like a rather dreary state of affairs -- but it has good cause.

Consider, for example, the "backwards" sequence on a list given as:

\begin{code}
backquence :: Applicative f => [f a] -> f [a]
backquence (x:xs) = flip (:) <$> backverse xs <*> x
backquence [] = pure []
\end{code}

This is a valid traversal, and is infinite productive on, for example, \hs{Reader} and \hs{Writer}. On \hs{State} it will bottom out on an infinite list. This corresponds to the fact that \hs{State} with a \hs{get} that actually can access the (undelayed) state, is not in fact predictable. The simulation of \hs{State} that is predictable simulates "get" through a function that is not bisimulation-invariant except in the circumstance that a traversal is prompt -- precisely what is violated here.

We define a prompt traversal, syntactically, as one in which every \hs{predict} is associated to the far right of an applicative chain. Consider the guarded-recursive version of backquence, as given below:

\begin{code}
ibackquence :: (Applicative f, Predictable f) => Stream (f a) -> f (Stream a)
ibackquence = lfix $ \rec x -> case x of
  Nil -> pure Nil
  Cons a s -> flip Cons <$> predict (rec <*> s) <*> a
\end{code}

The \hs{predict} call occurs in a term to the left of an \hs{<*>} where the right hand does not contain a predict call. So this traversal is not prompt. Conceptually, we have secretly been working with a form of "Later-bias" in our monoidal functors. Predictable monoidal functors that contain a product with a stable type use that stable type to "contain" \hs{Later} within them. The monoidal nature of applicative functors typically descends to a monoidal action on that type. The monoids we have considered thus far have been, as is typical, "left-strict-biased" -- this is to say that given (x <> y), the first non-later information will be nested in as many laters as induced by the \emph{left}-hand side of the append. Because predict calls force another layer of later-nesting, placing them on the left side of applicative application in a sense "inverts causality," and in particular disrupts the ability of non-bisimulation-invariant functions to access an undelayed first piece of information from our carrier monoid.

this means that there is some "inversion" of causality 
The \hs{predict} makes a best-effort to patch-up the mismatch of \hs{Later} levels. However, when an applicative action is not bisimulation-invariant, it is able to detect if it is operating on something with an additional \hs{Later} "swallowed" by such a \hs{predict}, and vary its behavior. So this ability to "work around" the inverted causality 


% NOT TRUE -- can still run forwards and backwards. just weird -- or can we? i.e. does it break the traversal laws for e.g. state, or writer and last? how does naturality intersect with taking a list then last -- try backwards traversal with traced state

% In particular, there are a limited set of infinite traversals that are "optimal" -- introduce as few laters as possible. Need a theory regarding gwbeq -- functors descend naturally when they respect gwbeq

% A consequence of the analysis of traversals in [Forwards and Backwards] is that traversals of a datatype correspond to permutations of its elements and in fact there are as many traversals of a given datatype as there are permutations on its arities. Despite that fact that traversable datatypes are also infinite-traversable, there are nonetheless far fewer infinite traversals on a datatype. For example, despite there being infinitely many finitary traversals of \hs{List}, there is precisely a single infinite traversal on \hs{Stream}. Infinite traversals necessarily fix visiting elements under fewer \hs{Later}s before visiting elements under more \hs{Later}s -- however, they nonetheless may permute their choices of which elements under any fixed quantity of \hs{Later} they visit first. So, there are multiple infinite traversals of \hs{ITree}, since at each step, there is a choice of which branch to traverse first, but nonetheless, these correspond only to the breadth-first, and not depth-first traversals, and so are only a subset of the finite traversals of \hs{Tree}.

A natural question to ask is if there is a single structure that can play the role with infinite traversals that lists do with finite traversals. In particular, finite traversable functors can be decomposed into a "structure" signature, and a "contents" list. If one swaps out "list" for something else (say, a stream, or a stream of lists), is the same possible for infinite traversals? Below, we consider an example that we think renders such a result very unlikely. It also serves as a good illustration of the flexibility of guarded recursive reasoning, as well as the subtleties of infinite traversals.

We sketch a standard "bi-infinite" stream, which can be "consed" (or "snoced") onto both ends. Such a structure, when built with Haskell lists, can be traversed with either the \hs{First} or \hs{Last} monoid, and depending on how it is built, possibly yield a non-bottom answer.

\begin{code}
data Bistream a = Bistream (Stream a) (Stream a) deriving Show

bicons x (Bistream xs ys) = Bistream (Cons x (Later xs)) ys

bisnoc y (Bistream xs ys) = Bistream xs (Cons y (Later ys))

backsequence :: (Applicative f, Predictable f) => Stream (f a) -> f (Stream a)
backsequence = lfix $ \rec x -> case x of
  Nil -> pure Nil
  Cons a s -> flip Cons <$> predict (rec <*> s) <*> a

instance ITraversable Bistream where
  isequence (Bistream x y) = Bistream <$> isequence x <*> backsequence y
\end{code}
  
\begin{comment}
\begin{code}    

data DLast a = DLast {getDLast :: Delay (Maybe a)} deriving Show

instance Stable (DLast a) where
  wait x = DLast (wait . fmap getDLast $ x)

instance Semigroup (DLast a) where
  x <> (DLast (Now Nothing)) = x
  x <> y@(DLast (Now (Just _))) = y
  x <> (DLast (Wait y)) = wait $ fmap (x  <>) (fmap DLast y)
  
instance Monoid (DLast a) where
  mempty = DLast (Now Nothing)

data DFirst a = DFirst {getDFirst :: Delay (Maybe a)} deriving Show

instance Stable (DFirst a) where
  wait x = DFirst (wait . fmap getDFirst $ x)

instance Semigroup (DFirst a) where
  (DFirst (Now Nothing)) <> x = x
  y@(DFirst (Now (Just _))) <> x = y
  (DFirst (Wait y)) <> x = wait $ fmap (<> x) (fmap DFirst y)
  
instance Monoid (DFirst a) where
  mempty = DFirst (Now Nothing)


z :: [Writer (DLast Int) Int]  
z = [tellit 1, tellit 2, tellit 3]
  where tellit x = tell (DLast . Now $ (Just x)) >> pure x
  
zx = Bistream (listToStream  (cycle z)) (listToStream (cycle $ drop 1 z))  

zz :: [Writer (DFirst Int) Int]  
zz = [tellit 17, tellit 1, tellit 2, tellit 3]
  where tellit x = tell (DFirst . Now $ (Just x)) >> pure x
  
zzx = Bistream (listToStream  (cycle zz)) (listToStream (cycle $ drop 1 zz))  

zzz :: [Writer (PStream Int) Int]
zzz = [tellit 1, tellit 2, tellit 3]
  where tellit x = tell (PCons x PNil) >> pure x



deriving instance Show a => Show (Delay a)
deriving instance Functor Stream

  
quux :: [State Int Int]
quux = [get]


\end{code}
\end{comment}

\section{Related Work}

\section{Future Work and Conclusion}

% Did we reinvent logical relations but shittier?


\end{document}

\begin{comment}

deriving instance Functor Delay
instance Applicative Delay where
  pure = Now
  Now f <*> Now x = Now (f x)
  f <*> Wait x = Wait $ fmap (f <*>) x
  Wait f <*> x = Wait (fmap (<*> x) f)


instance Predictable (Compose Delay Last) where  
  predict x = Later <$> Compose (wait $ fmap getCompose x)

  
instance Predictable Delay where
  predict x = fmap Later $ wait x

\end{comment}